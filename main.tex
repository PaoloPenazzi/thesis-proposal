\documentclass[12pt, a4paper]{article}
\usepackage[utf8]{inputenc}
\usepackage{fontenc}
\usepackage{xcolor}
\usepackage{hyperref}
\usepackage[english]{babel}
\usepackage[inline]{enumitem}
\usepackage{graphicx}
\usepackage{cleveref}

\graphicspath{ {img/} }

\newcommand{\versionmajor}{0}
\newcommand{\versionminor}{1}
\newcommand{\versionpatch}{0}
\newcommand{\version}{\versionmajor.\versionminor.\versionpatch}

\title{Progetto di Ricerca}
\author{Paolo Penazzi}

\begin{document}

\maketitle
% OR
% % ! TeX root = main.tex
\title{Title}
\author{Candidate Name Here}
\date{\today}

\newgeometry{margin=0.8in}
\begin{titlepage}
	\begin{center}
		% \vspace*{0.2cm}
		
		\large
		\textbf{ALMA MATER STUDIORUM -- UNIVERSITÀ DI BOLOGNA \\ CAMPUS DI CESENA}
		\\
		\noindent\hrulefill
		\vspace{0.4cm}
		
		\Large
		Scuola di Ingegneria e Architettura \\
		Corso di Laurea Magistrale in Ingegneria e Scienze Informatiche
		
		\Huge
		\vspace{4cm}
		\textbf{
			Title
			\\
			Subtitle
			\\
		}
		
		\large
		\vspace{1cm}
		Elaborato in 
		\\
		\textsc{Course}
		
		\vspace{5.5cm}
		\textit{Autori} 
				\\ 
				\textbf{Author 1}
				\\
				\textbf{Author 2}
		\vfill
		\noindent\hrulefill
		\vspace{0.3cm}
		\Large
		\\
		Anno Accademico 2022-2023
	\end{center}
\end{titlepage}
\restoregeometry

\section{Contesto e Motivazioni} \label{sec:context}

\section{Descrizione del progetto}

\section{Pianificazione delle attività}

\section{Periodo all'estero}

\paragraph{Ente ospitante}
L'ente ospitante è il \textit{Department of Electrical and Computer Engineering} dell'Università di Aarhus, Danimarca.

\paragraph{Motivazioni}
Il dipartimento ospitante è da tempo coinvolto in ricerche su questo ambito e rappresenta
quindi una grande opportunità per entrare in contatto con esperti del settore.
Ciò permette inoltre di basare il progetto di ricerca sulle esperienze maturate in passato
dai membri del gruppo di ricerca.

Inoltre lo spirito è quello di rafforzare un legame esistente tra i gruppi di ricerca che
condividono già molto sia per motivazioni che spirito e direzione di ricerca.

\paragraph{Supervisione}
Il lavoro di tesi verrà supervisionato da:
- Prof. Danilo Pianini, Università di Bologna (danilo.pianini@unibo.it) in qualità di relatore.
- Prof. Lukas Esterle, Aarhus University (lukas.esterle@ece.au.dk) in qualità di supervisore per l'ente ospitante.

\paragraph{Periodo}
Lo scambio, finalizzato alla preparazione della tesi magistrale avrà luogo da Ottobre a fine Gennaio.

%----------------------------------------------------------------------------------------
% BIBLIOGRAPHY
%----------------------------------------------------------------------------------------

%\nocite{*}
\bibliographystyle{plain}
\bibliography{bibliography}


\end{document}
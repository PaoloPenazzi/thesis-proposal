\documentclass[12pt, a4paper]{article}
\usepackage[utf8]{inputenc}
\usepackage{fontenc}
\usepackage{xcolor}
\usepackage{hyperref}
\usepackage[english]{babel}
\usepackage[inline]{enumitem}
\usepackage{graphicx}
\usepackage{cleveref}

\graphicspath{ {img/} }

\newcommand{\versionmajor}{0}
\newcommand{\versionminor}{1}
\newcommand{\versionpatch}{0}
\newcommand{\version}{\versionmajor.\versionminor.\versionpatch}

%\title{Progetto di Ricerca}
%\author{Paolo Penazzi}

\begin{document}

%\maketitle
% OR
% ! TeX root = main.tex
\title{Title}
\author{Candidate Name Here}
\date{\today}

\newgeometry{margin=0.8in}
\begin{titlepage}
	\begin{center}
		% \vspace*{0.2cm}
		
		\large
		\textbf{ALMA MATER STUDIORUM -- UNIVERSITÀ DI BOLOGNA \\ CAMPUS DI CESENA}
		\\
		\noindent\hrulefill
		\vspace{0.4cm}
		
		\Large
		Scuola di Ingegneria e Architettura \\
		Corso di Laurea Magistrale in Ingegneria e Scienze Informatiche
		
		\Huge
		\vspace{4cm}
		\textbf{
			Title
			\\
			Subtitle
			\\
		}
		
		\large
		\vspace{1cm}
		Elaborato in 
		\\
		\textsc{Course}
		
		\vspace{5.5cm}
		\textit{Autori} 
				\\ 
				\textbf{Author 1}
				\\
				\textbf{Author 2}
		\vfill
		\noindent\hrulefill
		\vspace{0.3cm}
		\Large
		\\
		Anno Accademico 2022-2023
	\end{center}
\end{titlepage}
\restoregeometry

\section{Contesto e Motivazioni} \label{sec:context}

Un problema chiave nell'esplorazione scientifica sono il testing e la validazione di approcci diversi allo stesso problema.
Nel campo dell'informatica questo problema diventa evidente quando si ha la necessità di replicare esperimenti
per confrontare o validare soluzioni differenti.
Spesso ciò che viene prodotto non viene mantenuto o semplicemente invecchia, rendendolo incompatibile con le nuove versioni
di sistemi operativi, librerie, etc.
Questo inevitabilmente porta a soluzioni `one-of` ai problemi senza una valida cross-validazione o, ancora peggio,
l'impossibilità di replicare i risultati.
Una soluzione a questo problema è l'uso di benchmark, ovvero di test che permettono di confrontare le prestazioni di
diverse soluzioni allo stesso problema.
La necessità di avere un maggior numero e una maggior qualità di benchmarks può essere osservata in vari campi dell'informatica.

\section{Descrizione del progetto}

\paragraph{Obiettivo}

L'obiettivo del progetto è la progettazione e l'implementazione di una piattaforma di benchmarking per CAS (Collective Adaptive Systems).

\paragraph{Pianificazione}

La pianificazione per raggiungere l'obiettivo del progetto è la seguente:
\begin{itemize}
    \item studio dello stato dell'arte e delle tecnologie (in particolare il lavoro svolto dal gruppo di ricerca di Aarhus);
    \item analisi dei requisiti e progettazione della piattaforma;
    \item sviluppo della piattaforma;
    \item test e validazione della piattaforma;
    \item stesura della tesi.
\end{itemize}

\section{Periodo all'estero}

\paragraph{Ente ospitante}
L'ente ospitante è il \textit{Department of Electrical and Computer Engineering} dell'Università di Aarhus, Danimarca.

\paragraph{Motivazioni}
Il dipartimento ospitante è da tempo coinvolto in ricerche su questo ambito e rappresenta
quindi una grande opportunità per entrare in contatto con esperti del settore.
Ciò permette inoltre di basare il progetto di ricerca sulle esperienze maturate in passato
dai membri del gruppo di ricerca e in particolare di continuare il lavoro svolto da essi svolto.

Inoltre lo spirito è quello di rafforzare un legame esistente tra i gruppi di ricerca che
condividono già molto sia per motivazioni che spirito e direzione di ricerca.

\paragraph{Supervisione}
Il lavoro di tesi verrà supervisionato da:

- Prof. Danilo Pianini, Università di Bologna (danilo.pianini@unibo.it) in qualità di relatore.
- Prof. Lukas Esterle, Aarhus University (lukas.esterle@ece.au.dk) in qualità di supervisore per l'ente ospitante.

\paragraph{Periodo}
Lo scambio, finalizzato alla preparazione della tesi magistrale avrà luogo da Ottobre a fine Gennaio.

\end{document}